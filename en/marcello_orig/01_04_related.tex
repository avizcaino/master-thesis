\section{Related Work} \label{sec:related_work}

Co-existence among wireless technologies has become an interesting problem because of the increasing density of the deployment of the various kind of wireless networks. A homogeneous case of coexisting networks is discussed in \cite{WLAN-loc-est}, where the authors consider a typical scenario of multiple overlapping \acp{WLAN}. They propose a new protocol, with lower overhead than \ac{RTS/CTS} schema, to mitigate the \textit{staggered} collisions, that happen when a device either is interrupted by or it interrupts an hidden node transmission, by evaluating their probability locally. The traffic state of the medium is modelled at each node by combining the node sensing results with information received with AP broadcasts and the collision probabilities are evaluated using a hard-coded table.

Different issues are faced when dealing with heterogeneous wireless networks. Because of the low transmission power, \acp{WSN} are particularly vulnerable to the interference introduced by other wireless technologies. Focusing on \acs{IEEE} 802.15.4 devices, the authors in \cite{Adapt-CCA} address the problem of reducing the inhibition loss only, that is the frame dropping caused by subsequent channel access failures. Since the nodes cannot fairly compete for spectrum access due to longer back-off periods with respect to \acs{IEEE} 802.11b/g devices, the authors propose an algorithm that adaptively raises and reduces their energy detection threshold, so the WSN nodes attemp to transmit even in presence of interference.

Furthermore, the solutions for \ac{WSN} co-existence with other more powerful technologies needs to account for the constraints of the \ac{WSN} nodes, i.\,e. low computational resources or energy efficiency. In \cite{Chan-ranking}, the authors developed a low complexity channel ranking algorithm to provide at the sensor network initialization phase a list of best channels used for transmission channel selection.

Instead of using a rarely refreshed channel list, a simple interference estimation mechanism to detect \acs{IEEE} 802.15.4 channels that overlap with used WiFi channels has been proposed in \cite{WiFi-Interf}. Before each multihop transmission, all the nodes on the path between source and destination sense the spectrum and then they coordinate to select the least noisy common channel to provide a fast and reliable virtual circuit, used for download operation.

The previous methods focus on the noise detection and avoidance, while in \cite{Interf-Class} the authors propose an algorithm that first \textit{classifies} the interference and then it adapts the \ac{WSN} nodes transmission protocol accordingly. The main considered sources of interference in a typical office or residential scenario are \acp{WLAN} and microwave ovens, thus the proposed method exploit the differences in the operating patterns (random access vs. fixed duty cycle) and then it interleaves the WSN communication in the duty cycles of a microwave device or it occupies the back-off of \ac{WLAN} devices by forcing the access to the spectrum before the \acs{DIFS}\footnote{\acf{DIFS}: defined by \acs{IEEE} 802.11 as idle the time that the \ac{WLAN} stations need to sense and wait before any transmission.} expires.

The scenario in \cite{Interf-Class} assumes that \ac{WSN} nodes have full knowledge of the availability of all the channels, while the \acs{POMDP} framework described in \cite{POMDP} introduces the concept of \textit{partial} knowledge due to the hardware demanding and energy inefficient sensing process. Thus, that framework adopts a decision-theoretic approach for spectrum sensing and channel access, with an underlying Markov process built on the partial observations of the spectrum activities.

In all those works, the authors provide methods for reacting to a discovered static or dynamic interference. However, by knowing the \ac{WLAN} spectrum activity pattern or being able to forecast it, the \ac{WSN} nodes may increase the communication quality in terms of throughput and energy saving. That knowledge may be acquired by analysing the \ac{WLAN} traffic and representing the trends of active and idle periods as probability distributions.

In \cite{Campus-WLAN}, the authors model with different granularities (system-wide or at \ac{AP}-level) many aspects for the workload of a whole campus \ac{WLAN}. Specifically, they describe at different level the traffic generation pattern, from the session arrivals to the flow sizes, using different distribution and estimating the parameters of those distributions through \ac{MLE}. Although their models are accurate and detailed, they do not go down to the spectrum occupancy level, not giving a model for idle periods that are the opportunities for \ac{WSN} transmissions.

Different probability distributions for describing the spectrum idle periods have been proposed in \cite{Spect-Opport}. The authors employ \ac{MLE} or \ac{EM} algorithms to estimate the parameters that best fit the distribution with traffic coming from real traces of an environment with heterogeneous wireless devices, and they show that the hyper-exponential distribution is an excellent candidate.

In \cite{DSA-Emp}, on the other hand, the authors give a mixture distribution that better models the \ac{WLAN} idle period distribution, since it captures the two basic sources of inactivity, that are short (almost uniformly distributed) back-off times or longer (heavy tailed) idle periods. They also propose an \ac{MLE} approach for estimating the parameters of the truncated distribution without the back-off time component. The distribution lacks the \textit{local view} perspective, essential when dealing with hardware-constrained sensors that have a limited detecting range.

Another \ac{WLAN} traffic estimation model has been proposed in \cite{ZigBee} where the authors introduce the concept of \textit{blindness} of the \ac{WLAN} with respect the \ac{WSN}. In \cite{ZigBee}, the authors describe and validate a model based instead on a Pareto distribution fitted through \ac{MLE} for describing the \ac{WLAN} activity trend and then they propose the frame control protocol \acs{WISE} that adapts the frame length for reducing the collision probability with the \ac{WLAN} transmissions.

In this work we will rely on the same \textit{blindness} concept but we will use the mixture distribution proposed in \cite{DSA-Emp}, since it covers realistic wireless scenarios and combine the two different sources of idle periods. We also emphasize the necessity of a locality in the traffic view due to the sensor limitations, that is missing in both of the previous quoted works.
