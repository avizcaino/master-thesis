\chapter{Introduction} \label{sec:introduction}

The proliferation of new wireless communication systems had lead to a scarcity of the wireless spectrum. However, as some studies have shown the spectrum is still lightly used \cite{gts1}-\cite{gts3}. This, in addition with the increase in the use of heterogeneous networks, leads to a necessity of new schemes to dynamic reuse the spectrum and take more profit of the resources available by introducing secondary systems. This increase in the usage of the spectrum will need to be controlled in order not to cause interference in the licensee (primary user) of the network. Mutual interference can be avoided in the design of the secondary system by using orthogonality in space, frequency or time \cite{gts3}. In this document we will approach a solution based on orthogonality in time, in which the secondary system will use the white spaces from the communication of the first system users.

In this project we will focus on \acs{WLAN} users as primary system users and Wireless Sensor Networks as secondary system. The interference applied by the primary system over the second one is considerable and the transmission of the secondary system can be neglected by these interferences. On the other hand, it is not necessary to take into account the interferences of the secondary system to the primary. In order to take profit from the white spaces in the spectrum left from the communication of the first system users and avoid the interferences, is necessary to predict when there will be a white space that can be used for the secondary system transmission. For this, it is necessary to study the behaviour of the users of the primary system. The solution under study in this project is the one introduced in \cite{gts1} and then extended in \cite{gts2}, \cite{gts3} and \cite{gts4}. These papers introduce as a solution a semi-markovian model which presents a equilibrium between complexity and accuracy. However, their solution is applied in scenarios with low load of WLAN users. Our approach is to study if this solution is applicable to larger scenarios. In \cite{hkps1} the session and flow layers of large scenario and traffic workload in a Campus WLAN are studied. In \cite{gfpp1} and \cite{lgfp1} the previous model is extended to packet level by applying the semi-markovian model introduced previously. In this project, we will focus in the developing of a set of tests to study if the model proposed is applicable to different types of scenarios.

\%\%\%\%\%\%\% Faltaria nombrar que en un dels papers 'gts' la cadena semi-markov, al IDLE model, no es t en consideraci CW i WS com diferents. Als segents papers aix s solucionat.\%\%\%\%\%\%\%


\%\%\%\%\%\%\% Faltaria nombrar que en un dels papers 'gts' la cadena semi-markov, al IDLE model, no es t en consideraci CW i WS com diferents. Als segents papers aix s solucionat.\%\%\%\%\%\%\%


\%\%\%\%\%\%\% Faltaria nombrar que en un dels papers 'gts' la cadena semi-markov, al IDLE model, no es t en consideraci CW i WS com diferents. Als segents papers aix s solucionat.\%\%\%\%\%\%\%


\%\%\%\%\%\%\% Faltaria nombrar que en un dels papers 'gts' la cadena semi-markov, al IDLE model, no es t en consideraci CW i WS com diferents. Als segents papers aix s solucionat.\%\%\%\%\%\%\%


\%\%\%\%\%\%\% Faltaria nombrar que en un dels papers 'gts' la cadena semi-markov, al IDLE model, no es t en consideraci CW i WS com diferents. Als segents papers aix s solucionat.\%\%\%\%\%\%\%