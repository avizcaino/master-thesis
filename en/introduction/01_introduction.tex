\chapter{Introduction} \label{chapter:introduction}
The proliferation of new wireless communication systems had lead to a over-crowded unlicensed band, limiting the performance of the different systems due to the high interference environment. However, as some studies have shown, the spectrum is still lightly used \cite{ActiveMeasure}-\cite{CMA-Exp} at most times and locations. This, in addition with the increase in the deployment of heterogeneous networks, leads to a necessity of new schemes to dynamically reuse the spectrum and increase the utilization of the resources available by introducing secondary systems. This increase in the usage of the spectrum will need to be controlled in order not to cause interference in the licensee (primary user) of the network. Mutual interference can be avoided in the design of the secondary system by using orthogonality in space, frequency or time \cite{CMA-Exp}. In this document we will approach a solution based on orthogonality in time, in which the secondary system will take profit of the white spaces left from the communication of the primary system users.

In this project we will focus on \acs{WLAN} users as primary system users and Wireless Sensor Networks \acs{WSN} as secondary systems. The interference applied by the primary system over the second one is considerable and the transmission of the secondary system can be damaged by this interference. On the other hand, it is not necessary to take into account the interference of the secondary system to the primary since the transmit power of the primary user is much higher than the secondary user. The primary system communication leaves white spaces between active periods. In order to take profit from these white spaces in the spectrum left from the communication of the first system users and avoid the interference, it is necessary to predict when there will be a white space that can be used for the secondary system transmission and how long it will last. The interference of the primary system will have a high effect over the secondary system transmissions, which will cause a high number of collisions and therefore, force to retransmit the packets in order to complete the transmission. This last will affect the energy efficiency of the system, which is a critical issue due to the resource limitations in the \acs{WSN} technology.

The solution under study in this project is the one introduced in \cite{ActiveMeasure} and then extended in \cite{DSA-Emp} and \cite{CMA-Exp}. These studies introduce as a solution a semi-Markovian model which presents a balance between complexity and accuracy, in order to model the \acs{WLAN} traffic. However, their solution is applied in scenarios with low load of \acs{WLAN} users. Our contribution is to study if and when the previous solution is applicable to different scenarios in order to model the traffic.

In order to model the traffic, we will use the model presented in \cite{Campus-WLAN} which develops a study of the session and flow levels of larger scenarios and traffic workload in a campus \acs{WLAN}. In \cite{marcello} the previous model is extended with the packet level by applying the semi-Markovian model introduced previously, conforming the final multi-layer traffic model that will be used in this project.

In this project, we will focus in the developing of a set of tests to study the fitness of the semi-Markovian model to model the \acs{WLAN} traffic using a multi-layer traffic model as the one presented in the previous papers for different types of scenarios.