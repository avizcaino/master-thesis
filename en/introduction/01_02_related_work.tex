\section{Related Work on WLAN Spectrum Usage Modeling} \label{sec:intro_related}
Different studies have been developed around the issues that spectrum sharing in time-domain exploiting the white-space \acs{WS} between transmissions in a \acs{WLAN} 802.11 based network. In order to model the \acs{WLAN} traffic, different proposals had been presented.

In \cite{ActiveMeasure}, a first proposal of a continuous-time semi-Markovian model is presented in order to model the \acs{WLAN}'s channel behavior. This model presents a good trade-off between analytical tractability and accuracy. The aim is to implement a capability to sensors in order to predict the primary user's behavior by sensing the traffic generated by this primary user and later on generate a model using the proposed semi-Markovian model.

Due to the heavy-tail behavior of the idle periods, the continuous-time Markov process is not the proper model since the sojourn times in each state should be exponentially distributed. Because of this, they propose a semi-Markov model which allows an arbitrary specification of the sojourn time in each state. The sequence of states of transmission of Data - SIFS - ACK follows a deterministic behavior, which they decide to include them together as a single Active state. On the other hand, the white spaces between transmissions are defined as Idle state. Because the Idle state follows a heavy-tailed behavior, they consider a Generalized Pareto distribution to model this state. Even though, some problems appear with the Generalized Pareto distribution for high load scenarios.

Due to the problems presented in \cite{ActiveMeasure} because of the bad fitting for high load scenarios, in \cite{DSA-Emp} the previous semi-Markovian model is extended by modifying the Idle distribution previously defined. In \cite{ActiveMeasure} is just considered that the idle periods are determined by the white-spaces of the transmissions. On the other hand, the contention window (\acs{CW}) should be included also in the Idle state. For this, a mixture idle distribution to model the Idle periods of the channel is proposed. The mixture idle distribution consists in a combination of two distributions that model each the \acs{CW} and \acs{WS}. Since the \acs{CW} follows an uniform behavior, an Uniform distribution is proposed as a solution. On the other hand, for the heavy-tail behavior of the \acs{WS}, the Generalized Pareto is still a good solution to model this behavior but in this case, this distribution will be left-truncated due to the inclusion of the \acs{CW} in the model.

In \cite{Campus-WLAN}, the traffic of a Campus \acs{WLAN} is studied in order to find the proper distributions to model it. Proposing a multi-layer traffic model where two different layers or levels of traffic are determined: session and flow. The packet level is not studied. In order to model each one of the levels, different distributions are proposed that fit the behaviour of each one. This multi-layer traffic model is extended with a new level for the packets in \cite{marcello}. In addition, the 2-state semi-Markovian model is extended to a 3-state for the partially-observable model for the sensor nodes. 

In \cite{CMA-Exp}, using as a base the model presented in \cite{ActiveMeasure} and \cite{DSA-Emp}, a Cognitive Medium Access (\acs{CMA}) is introduced in order to be implemented over sensors to coexist with the \acs{WLAN}. Two models are presented differentiated by the observable capability of the channel: full-observable and partially-observable, which will be a starting point in the two models presented in \cite{ioannis}.

In \cite{ioannis} a Cognitive Access Scheme is presented for \acs{WSN}s that coexist with \acs{WLAN}s, considering the blind and hidden \acs{WLAN} terminals, in order to decrease the negative effect of the coexistence problem and normalize the energy cost considering the limited sensing capabilities of the sensor nodes. The mixture idle distribution proposed in \cite{DSA-Emp} is used to model channel occupancy. The cognitive capabilities designed for the sensor nodes include: capability to differentiate if an idle period is due to \acs{CW} or \acs{WS}, decision capacity over the packet size and next hop distance, and predict whether there is sufficient \acs{WLAN} idle time for the transmission of a \acs{WSN} packet. The final results show that the proposed solution achieves a significant gain in performance compared to the traditional channel access solutions for typical \acs{WLAN} load values.

In our project, the main goal will be extending the work developed in \cite{marcello}, designing a set of experiments to test the different implementations done and correct or extend when is needed. Also a set of experiments will be developed in order to test the complete semi-Markovian multi-layer traffic model for different traffic realizations in order to find in which situations the model is suitable.