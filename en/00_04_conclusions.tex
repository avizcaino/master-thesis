\chapter{Conclusions} \label{chapter:conclusions}

An extended set of tests has been developed for this project in order to determine whether and when the proposed model for the spectrum activity is suitable to be used.

As it has been explained, both idealistic and realistic approaches to model the observable load of the sensors are represented by semi-Markovian processes and the sojourn times in the Idle and Active states are defined by some determined distributions. The first step in this project was to test the validity of the proposed distributions for both states. For this, it has been proved that to model the holding times in the idle state, the proposed mixture distribution suits, showing a clear different behaviour in the distribution of the idle periods, in both idealistic and realistic models under study.

The second step was to test the estimation and validation processes in both approaches for different traffic scenarios. First of all, in the Global View model, the estimation process showed to be insensitive to the randomization of the packet level, showing that estimated parameters differed with low deviation for multiple runs of the same configuration of the higher levels (session and flow) in the multi-layer traffic model. In addition, the process was also insensitive for different number of sessions. For the validation process, the main requirement is that the active and idle distributions should be perfectly reconstructed using the estimated parameters. In order to test the quality of the reconstructed distributions and, in extension, the validity of the proposed model, a validation test such as the Kolmogorov-Smirnov test has been used. Different problems had been faced in this section of the thesis. At the beginning, the validation test chosen didn't show up to be a good tool in order to determine the quality of the reconstructed distributions. Visually, the fitting between both empirical and estimated distributions was perfect, but applying the validation test, this presented a high failure-rate. Different improvements have been implemented in the test, showing high improvement when applying the test on the truncated part of the idle distribution, avoiding applying the validation test in the uniformly distributed part of the idle distribution. Also, we presented that there is an improvement when using a high number of samples for the validation test. The results in the \acs{K-S} test presented that the validation test is also insensitive for different number session cases. Finally, we wanted to determine a load region in which the proposed model can be applied with high success. The results presented that the proposed model can be applied perfectly the load region of 10 to 30 \% of load, presenting a low failure rate in the validation test in addition to be insensitive to the session number.

Finally, a similar set of experiments was developed for the realistic approach (Local View). Since the mixture idle distribution followed the same behaviour in this realistic model, we proceeded with the estimation and validation tests. First of all, an analysis of the active periods was developed in order to prove the assumption in the Local View model in which is stated that the consecutive active periods are generated by totally independent sources. In addition, different validation tests were applied for this model such as Mean-Square error and Chi-squared but none of these were a good solution. In the Local View model, we implemented the estimation process using the Laplace Transform. A qualitative analysis of the local estimated parameters has been done comparing the results of the Laplace estimation process with the estimated parameters in the Global View model. The results of the experiments in this section showed that the estimation process is highly affected by the load and the number of sessions, presenting estimated parameters closer to the ones in the Global View model when the load is higher for a same number of sessions. On the other hand, a higher number of sessions also presented a small improvement in the estimated parameters.

\section{Future Work}
Multiple experiments in both approaches had been developed for this thesis. However, as the results have shown, there are still different issues to be solved. First of all it is necessary to find a proper validation tool in order to test the quality of the reconstructed distributions in both Global and Local View models, although it is more critical for the second approach. Future work may also include an extended set of experiments to determine if the proposed model can be applied in a wider region of loads in addition to the 10 to 30 \% load region in which the model can be applied with high success. Finally, a deeper study of the Local View model is needed, developing tests for a more extended set of traffic scenarios in order to determine when the model is suitable to be used.