\pdfbookmark{Abstract}{Abstract}
\chapter*{Abstract}
\thispagestyle{begin}

% JOHN: I am working on the abstract, directly here.

%problem definition - different tech, same band, concurrent access. different power

The deployment of heterogeneous wireless networks in the same spectrum space introduces the need for dynamic spectrum access so as to increase the utilization of the available wireless resources. Dynamic spectrum access needs to be controlled in order avoid interference between the users of different systems. Different schemes can be used in order to avoid the mutual interference between the systems: orthogonality in space, frequency or time. In this thesis we address the problem with a solution based on time orthogonality, in which the coexisting wirless systems are a \acs{WLAN} and \acs{WSN}.

Due to the high power asymmetry it is necessary to implement a cognitive capability in the most affected system, i.e. the WSN, which will predict the behaviour of the WLAN spectrum usage and take advantage of the white spaces left for WSN interference-free communication. For this, it is necessary to model the traffic of the WLAN system. The applicability of two different semi-Markovian models has been studied in the scope of this thesis: one represents an ideal case in which the sensors have unlimited sensing capabilities and a second, more realistic, approach in which the sensor view is limited by hardware and resources. In this project we investigate whether and when the proposed models are suitable to be used in order to model, estimate and predict realistic \acs{WLAN} channel usage; for that we consider a measurement-based, multi-layer WLAN traffic workload model.

Different experiments have been developed to test different traffic scenarios in which we apply our prediction model. The experiments show that the WLAN usage estimation process is robust, i.e. insensitive to irregularities introduced by the packet level randomization and the underlying protocols in the WLAN. An almost perfect fitting is achieved in a wide range of cases between the distributions to model the active and idle periods and the empirically derived channel usage functions. In addition, we study different usage load regions in which we apply our model and the results show that it can be applied with high success in a region of 10 to 30 \% of load. On the other hand, the realistic model, based on partial observation of WLAN trafic, shows higher variations between different traffic conditions, increasing the performance of the estimation process in cases of higher WLAN load.

% Well, it is not perfect, but perhaps sufficient for the presentation next week. If you want to change things, feel free to do it.

\acresetall %acronims
%if you want to define an acronym, define with \acs{word}. Acronyms are defined in 99_acronyms
