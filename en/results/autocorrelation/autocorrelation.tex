\chapter{Autocorrelation Study} \label{chapter:autocorrelation_results}
As it has been explained in previous chapters (see Chapter \ref{chapter:local_view}), the sensors have a limited sensing capability due to hardware limitations. In the Global View case, we studied an ideal case in which the sensors can observe the traffic of the whole network. On the other hand, the Local View case, the sensing limitation is translated into a limited observation of the traffic of the network: the sensors will only be able to observe the traffic generated by the \acs{WLAN} users inside the sensor's sensing range.

In order to solve this, a three-state Semi-Markovian model has been proposed in \cite{marcello} (represented in Figure \ref{fig:semi-markov_local}) in which the observable and non-observable active periods are taken into account for the model using the $Pcca$ probability, which determines the amount of non-observable traffic. The assumption in this model is the fact that the consecutive active \acs{WLAN} periods are independent, i.e. are originated from independent \acs{WLAN} users or "sessions". 

So, before introducing the experiments to test the Local View model, it is necessary to prove this assumption. In this chapter, we will test the extended multi-layer traffic model in NSMiracle and determine under which conditions and to what extend this assumption is valid. We will study the autocorrelation sequence of the active periods of different traffic configurations and see whether the assumption of independency holds using the same procedures developed for the experiments of the Global View model, where we followed a systematic way when designing the experiments.

This chapter will present the results of the independence study as an introduction to the experiments for the Local View experiments.