\section{Methodology} \label{sec:autocorrelation_methodology}

\subsection{Experiments setup} \label{sec:experiment_setup}
A basic setup will be used for the experiments. Several configurations can be followed to perform the tests. We defined a scenario composed by a determined number of \acs{WLAN} users uniformly distributed and an Access Point with a coverage of 100 meters. The number of users and traffic configuration will be determined by the needs of each one of the experiments. We will use the extended multi-layer traffic model presented in \cite{Campus-WLAN} and \cite{marcello}.

In this case, the sensors have a limited sensing capability due to hardware limitations. Each sensor will have a sensing range of 35 meters using the sensor's characteristics used in the experiments performed in \cite{marcello}. Different sensors will be deployed on the network. The number will be determined by the needs of each one of the experiments.

For the autocorrelation study that we will perform in this section, we extracted the active periods of the system, but just taking into account the active periods generated by Data packets. We avoided the inclusion of the \acs{CTS}, \acs{RTS} and \acs{ACK} packets for the autocorrelation study.

\subsection{Autocorrelation experiments for the Global View model} \label{subsec:autocorrelation_active}
As it has been explained, one of the assumptions in the Local View model is that the different active periods in the network are independent since are generated by independent \acs{WLAN} users. We will test this assumption. For this, first of all we will study the active periods of the whole network (Global View model), in order to test whether this assumption holds or not. In case the assumption holds, then we will study the independence of the active samples gathered by the sensors in the Local View model, which have a limited sensing range.

In order to study this independence of the samples we will use the autocorrelation function. The autocorrelation function is the cross-correlation of a signal or sequence with itself. It is the similarity between observations as a function of the time separation between them. The autocorrelation gives us the independence between samples of a sequence. A autocorrelation function which approximates to a $\delta$ means that the samples in the sequence are high independent. On the other hand, a flat autocorrelation function means that the samples are not independent.

We will follow a similar procedure as the final tests developed for the Global View model: 

\begin{itemize}
	\item \textbf{Session Experiment}: First of all we will test the independence of the active periods for different number of \acs{WLAN} sessions, fixing the flow level and randomizing the packet size and packet interarrivals as did before. All the sessions will have a similar load so we can compare them.
	\item \textbf{In-session Experiment}: Secondly, we will study a single case with a determined number of sessions and we will randomize the flow and packet levels as we did in Section \ref{subsec:globalview_insession} which will represent different load cases.
\end{itemize}

%%\subsection{Autocorrelation of the Idle periods} \label{subsec:autocorrelation_idle}
%%In addition to the experiments performed for the active periods, we will perform more tests on the validation test. As it has been presented before, the validation test used in this project is the Kolmogorov-Smirnov. The \acs{K-S} requires that the samples used for the test need to be independent. The samples in this case are the idle periods. Then, we will study the independence of these samples and make, if necessary, the proper modifications to the test in order to try to achieve a better performance than the presented in Section \ref{sec:ks-results}.

%%In Section \ref{sec:ks-results} it has been shown that the results of this validation test are not good enough and we need to reconsider the use of the \acs{K-S} as a validation test for the fitting of empirical and estimated distributions. We will try to find a solution for this problem doing an intensive study of the independence of the idle samples used for the test.

\subsection{Autocorrelation experiments for the Local View model} \label{subsec:autocorrelation_lv}
Finally, we will study the autocorrelation of the samples gathered by different sensors as an introduction to the experiments we will develop for the Local View model.

In this case, we will place different sensors with a limited sensing range that will observe different portions of the network. In this section we are not performing estimation of the parameters. The sensors will extract a sequence of active samples that will be used for the autocorrelation study. We will compare the samples extracted by the sensors with the total amount of traffic in the network and generate a sequence of IN/OUT samples for each sensor in order to study the autocorrelation of this sequence. This sequence will present the samples that the sensor observed and the ones that are out of the sensing range. The results of this experiment will give us insight of whether the samples are independent enough so the semi-markovian model proposed for the Local View model can still be used.