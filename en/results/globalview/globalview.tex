\chapter{Global View: Results} \label{chapter:global_results}
This chapter presents the results of the experiments designed to test the validity of the Global View Model proposed in \cite{ioannis}. In order to test the model, a systematic way has been followed to analyse in a proper way the results. 

The experiments carried out in this part of the project consist of a deeper study of the Global View model that was the starting point in \cite{marcello} with the difference that the tests will be carried on the multi-layer traffic workload model presented in \cite{Campus-WLAN} and later extended in \cite{marcello}, in order to test the validity of the proposed model in real \acs{WLAN} traffic scenarios. In order to perform the tests, we will use the NS Miracle framework and the implementation of the Global View estimator developed in \cite{marcello}. Different tests will be carried out over the different levels of the multi-layer traffic model. For this, each one of the other levels that are not going to be tested should be fixed for all the runs of the same simulation configuration, changing a single layer at a time, in order to be able to analyse the final results properly. We will focus on the validation of the Global View model for the idle distribution.

The Global View validation process is composed by different phases: the sensors observe the \acs{WLAN} traffic and estimate the parameters needed in order to reconstruct the idle distribution, finally a validation test is started, which will test the fitting between the empirical idle distribution and the generated from the estimated parameters. We will carry a series of experiments to test each one of the phases and correct possible errors.
