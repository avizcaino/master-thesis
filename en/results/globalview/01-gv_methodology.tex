\section{Methodology} \label{sec:globalview_methodology}
As it has been explained before, the Global View validation can be divided in three phases. We will perform different experiments over each one of these three phases:

\begin{itemize}
	\item The sensor observe the \acs{WLAN} traffic and extract the idle distribution.
	\item The sensor then perform the estimation of the parameters needed in order to reconstruct the idle distribution.
	\item Finally, a validation test is executed in order to test the fitting between the empirical idle distribution and the one generated from the estimated parameters.
\end{itemize}

From the results of the experiments we will conclude if each one of the phases are working correctly and, if needed, we will extend and modify the estimation and validation processes to refine the work developed in \cite{marcello}.

\subsection{Scenario setup} \label{sec:sensor}
A basic setup will be used for the experiments. Several configurations can be followed to perform the tests. For the experiments developed to test the Global View model, we will define a scenario composed by a determined number of \acs{WLAN} users uniformly distributed and an Access Point with a coverage of 100 meters. The distribution of the users is not crucial for the Global View model since the sensor is capable to observe the whole traffic. The only condition is to avoid hidden terminals and a \acs{WLAN} node should be visible by all the other nodes. The \acs{WLAN} nodes and \acs{AP} will use the 802.11 \acs{IEEE} standard with a transmit rate of 11 Mbps and a transmit power of 15 dBm. The number of users will be determined by the needs of each one of the experiments. In addition, we will deploy a single sensor in the scenario. The sensor can be positioned anywhere since we are working with the Global View model. In our case, we will fix a sensor in the same position as the \acs{WLAN} \acs{AP}. 

The main characteristic of the Global View model is that the sensor can observe the whole spectrum activity, which means that the sensor will have the same sensing range than the \acs{WLAN} coverage range. We will use an ideal case in which the sensor will be observing the network from the start of the simulation, considering that all the users already arrived to the network and will gather a determined number of samples that will be used for the estimation process.

\subsection{Study of the extraction of the Idle Distribution} \label{sec:idle}
The first step before the estimation of the parameters is to reconstruct the idle distribution. It is necessary to test if the distribution generated from the idle samples extracted by the sensor, can be modelled using the Idle function defined in (\ref{eq:Idle}). As it has been presented in Chapter \ref{chapter:model}, the idle distribution can be approximated by a mixture distribution (see (\ref{eq:Idle})) which is formed by a uniform distribution to approximate the \acs{CW} in the range of $[0 < t < \alpha_{bk}]$, and a pareto distribution for the heavy-tail behavior of \acs{WS} in $[t > \alpha_{bk}]$. If the idle distribution does not follow this behaviour, the estimation process cannot be done.

Firstly, we will make a visual validation of the idle distribution. We will check that the idle periods generated by the simulator using the multi-layer traffic model can really be approximated by the mixture idle distribution. For this, we will extract the \acs{CDF} of the idle periods from the trace file generated by the simulator and we should be able to observe the uniform and the heavy-tailed behaviors in the specified areas.

In addition, we will test the effects of active distribution (packet sizes) on the idle behaviour. For this, we will test the sensitivity of the idle distribution against different packetization processes. The packetization process is determined by the packet sizes and their inter-arrival times within a flow, and the \acs{WLAN} transmit rate. For these experiments we will use different traffic configurations in the packet level, which means that in order to compare the results, we need to fix the values of the other levels of the multi-layer traffic model (session and flow levels). Each of the sessions will use a different number of flows and inter-arrival times.

We will test different active distributions (determined by the packet size) for the same idle distribution (determined by the inter-arrival times between packets) and compare the obtained results. If the process is carried correctly, the results should show that the idle distribution is almost insensitive to the active distribution changes.

\subsection{Estimation Process} \label{sec:MLE}
The second part of this chapter will test the estimation process of the proposed model. More in detail, will test the algorithm design and configuration developed in \cite{marcello}. The estimation process is carried by the sensors which, from the observations over the \acs{WLAN} spectrum activity, should be able to estimate the needed parameters for the mixture idle distribution in order to reconstruct the \acs{WLAN} spectrum activity and be able to predict its behavior.

As it is introduced before in this document and in \cite{ioannis} and \cite{marcello} the estimation process is carried out using Maximum Likelihood Estimation (\acs{MLE}). The correct functioning of this process is a key point of the model since these parameters will be used by the sensors in order to reconstruct the behavior of the network and predict when it will be idle and hence, be able to send data. The estimated parameters should be the same for the same high-level statistics.

The main goal of these experiments is to test whether or not the randomization of the packet level affects the estimation process, affecting the stability of the estimated parameters that will be used later to reconstruct the idle distribution.

In order to test the \acs{MLE}, the same simulation with a fixed traffic configuration will be run several times and the estimation parameters will be extracted in order to be studied later. The traffic configuration will be the same as the defined in Section \ref{sec:idle} in which the session and flow levels are fixed while the packet level is randomized. If the estimated parameters do not differ in a high manner, then the estimation process is insensitive to the packet level. We will compare different random distributions in the packet level in order to test the estimation process. From the estimated parameters of the different tests, the mean and standard deviation will be extracted.

\subsection{Model Validation} \label{sec:ks}
Once we have tested the correct functioning of the generation of the idle distribution and the estimation process, is necessary to check if the reconstructed idle distribution fits with the empirical one. For this, we will use a Goodness-of fitness test which is a validation test that will check both distributions and determine how good the proposed model is for the determined scenario. The implemented validation test is the Kolmogorov-Smirnov test. The \acs{K-S} test is a Goodness-to-fitness test that will test the fitting between the empirical distribution (obtained from the simulated traffic) and the estimated distribution constructed from the estimated parameters. This validation test measures the deviation (D-value) of the empirical and experimental functions and the probability of null-hypothesis (P-value). The scientific community had determined that a null-hypothesis will be rejected if ${P-value<0.05}$.

In order to prove that the \acs{K-S} test is a good Goodness-of-fitness test for the proposed model, different tests will be performed following the same procedure for testing the \acs{MLE} in Section \ref{sec:MLE}. Multiple runs of the same configuration will be carried and we will extract the D and P values of the \acs{K-S} in order to study the deviation between the parameters.

The experiments performed in this section will give us an insight of how optimal is the \acs{K-S} test for our model and if it is necessary to use another validation test.

%\subsection{Autocorrelation study of the Idle periods for the \acs{K-S} test} \label{sec:autocorrelation_idle}
%As it has been presented before, the validation test used in this project is the Kolmogorov-Smirnov. The \acs{K-S} requires that the samples used for the test need to be independent. The samples in this case, are the idle periods gathered for the estimation process. We will study the independence of these samples and make, if necessary, the proper modifications to the test in order to try to achieve a better performance.

%In order to study this independence of the samples we will use the autocorrelation study. The autocorrelation function is the cross-correlation of a signal or sequence with itself. It is the similarity between observations as a function of the time separation between them. The autocorrelation gives us the independence in time between samples of a sequence. A autocorrelation function which approximates to a $\delta$ means that the samples in the sequence are high independent. On the other hand, a flat autocorrelation function means that the samples are not independent.

\subsection{Effect of the number of samples in the Kolmogorov-Smirnov validation test} \label{sec:ks_optimization_presentation}
In this experiment we studied the impact of the number of samples in the Kolmogorov-Smirnov validation test. The number of idle samples used for the \acs{K-S} has an impact in the time of performance. The first implementation of the validation test for our model uses a 10\% of the idle samples gathered for the estimation of the idle-distribution parameters. This decision has been made in order to achieve the estimation of the p-value in the \acs{K-S} test faster. This experiment will give us an insight of the impact of using a percentage of the total set of samples in the performance of the validation test.

\subsection{Session and in-Session Experiments} \label{sec:final_gv}
Finally, after testing the mixture idle distribution, the estimation process and the validation test, it is necessary to test the combination of \acs{MLE} and \acs{K-S} validation test for a wide range of parameterizations of the multi-layer traffic workload model. Instead of full-randomization of the input variables, we will conduct detailed experiments for different areas of each traffic variable [e.g. session/flow number etc.], representing a different "operation points" of the \acs{WLAN} network. The objective is to test and identify the areas where our modelling fits better and where it fails.

Again, we will test each one of the three phases presented at the beginning of this section with similar procedures followed in each one of the previous sections.

The outcome of the following experiments will be a general evaluation of the designed Global View Model in different traffic scenarios. The results should give insight into possible extensions of the activity model that match better the considered traffic workload model.

\subsection{Autocorrelation study of the Active periods for the Global View model} \label{subsec:autocorrelation_active}
As it has been explained in Chapter \ref{chapter:model}, one of the assumptions in the Local View model is that the different active periods in the network are independent since are generated by independent \acs{WLAN} users. We will test this assumption. For this, first of all we will study the active periods of the whole network (Global View model), in order to test whether this assumption holds or not. In case the assumption holds, then, in following chapters we will study the independence of the active samples gathered by the sensors in the Local View model, which have a limited sensing range.

Here we will use again the autocorrelation function to study the independence of the active samples. We will use the following set of experiments:

\begin{itemize}
	\item \textbf{Session Experiment}: First of all we will test the independence of the active periods for different number of \acs{WLAN} sessions. All the cases will have a similar load so we can compare the effect of the number of sessions in the sequence of active periods.
	\item \textbf{In-session Experiment}: Secondly, we will study a single case with a determined number of sessions (i.e. 10 sessions) and different load cases for the same number of sessions. With this, we will study the effect of the load in the active periods sequence.
\end{itemize}