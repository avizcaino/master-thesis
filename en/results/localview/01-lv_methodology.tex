\section{Methodology} \label{sec:localview_methodology}
he Local View model can be divided in three phases. We will perform different experiments over each one of these three phases:

\begin{itemize}
	\item The sensor observe the \acs{WLAN} traffic and extract the idle distribution.
	\item The sensor then performs the estimation parameters needed in order to reconstruct the idle distribution.
	\item Finally, a validation test is executed in order to test the fitting between the empirical idle distribution and the generated from the estimated parameters.
\end{itemize}

From the results of the experiments we will see if each one of the phases are working correctly and, if needed, we will extend and modify the implementation to refine the work developed in \cite{marcello}.


\subsection{Scenario Setup} \label{subsec:localview_scenario_setup}
A basic setup will be used for the experiments. Several configurations can be followed to perform the tests. For the experiments developed to test the Local View model, we will define a scenario composed by a determined number of \acs{WLAN} users uniformly distributed and an Access Point with a coverage of 100 meters. The \acs{WLAN} nodes and \acs{AP} will use the 802.11 \acs{IEEE} standard with a transmit rate of 11 Mbps and a transmit power of 15 dBm. The number of users will be determined by the needs of each one of the experiments. In addition, we will fix different number of sessions depending on the needs of the experiments.

The sensors in this case, have a limited sensing range due to hardware limitations. In difference with the Global View model, in which the sensors are capable of observe the whole traffic of the network, in the Local View model, the sensors can only observe the traffic generated by the \acs{WLAN} users that are within the sensor's sensing range. This makes more difficult to reconstruct the spectrum activity since the sensors don't know what is happening in the rest of the network.

\subsection{Active Distribution} \label{subsec:localview_active}
In this section we will study the independence of the active periods in order to investigate the assumption in the 3-state semi-Markovian model, in which is stated that the consecutive active \acs{WLAN} periods are independent, i.e. are originated from independent \acs{WLAN} users or 'sessions'. 

We will study the autocorrelation sequence of the active periods of different traffic configurations of the Multi-layer traffic model and see whether the assumption of independence holds using the same procedures developed for the experiments of the Global View model, where we followed a systematic way when designing the tests.

In addition, we will study the sequence of consecutive skipped active periods due to the limited sensing range capabilities of the sensor. It has been proved that this sequence follows a geometric distribution. For this, we will test the fitting between the empirical and geometric distributions.

It is known that the Kolmogorov-Smirnov test is not a proper validation test for discrete distributions. For this, other alternatives such as Mean-Square Error and Chi-square validation will be tested.

This section will present the results of the independence study as an introduction to the experiments for the rest of the Local View experiments.

%\subsection{Estimation Process} \label{subsec:localview_estimation}
%The estimation process developed for the Local View model can be carried with a the Laplace Transform algorithm. The Laplace Transform method was briefly studied and implemented and we will complete the implementation in the NSMiracle framework and carry a full battery of experiments to test it.

%This section will present the implementation and different experiments developed to test the Laplace Transform method as a method for the estimation of the parameters in order to construct the idle distribution for the sensors.

%\subsection{Model Validation} \label{subsec:localview_validation}
%Repeating the process followed in the Chapter \ref{chapter:global_view_results}, we will design a set of tests to test the validation method (in this case also the Kolmogorov-Smirnov test) for the test of the fitting between empirical and estimated distributions.

\subsection{Session and Load Experiments} \label{subsec:localview_session}
Once the active periods independence and the consecutive skipped active periods has been studied, we will proceed with the final set of experiments for the Local View model.

For this section, a series of tests over the session level has been developed. Here, we will test the validity of the Local View model over different session and load cases, and how the estimation process is carried out.

These experiments will give us a validation of the results previously presented.